%%% Local Variables: 
%%% coding: utf-8
%%% mode: latex
%%% TeX-engine: xetex
%%% End:
\documentclass[a4paper,12pt]{article} 

%\usepackage[hmargin=2.25cm, vmargin=1.5cm]{geometry} % Document margins
\usepackage[right=2cm, left=2cm,bottom=2cm,top=2cm]{geometry}

\usepackage[usenames,dvipsnames]{xcolor} % Allows the definition of hex colors
\definecolor{linkcolor}{HTML}{506266} % Blue-gray color for links
\definecolor{shade}{HTML}{F5DD9D} % Peach color for the contact information box
\definecolor{text1}{HTML}{FF0000} % Main document font color, off-black
\definecolor{headings}{HTML}{701112} % Dark red color for headings
% Other color palettes: shade=B9D7D9 and linkcolor=A40000; shade=D4D7FE and linkcolor=FF0080

\usepackage{polyglossia}
\setmainlanguage[variant=uk]{english}

\usepackage{fontspec,xltxtra,xunicode}
\defaultfontfeatures{Ligatures=TeX}
\defaultfontfeatures{Mapping=tex-text}
\setromanfont[Mapping=tex-text,SmallCapsFeatures={Scale=1.1},Numbers=OldStyle]{Optima}%{Adobe Garamond Pro} %{Hoefler Text} % Main document font
\setmonofont[Scale=0.85]{Menlo}

\newfontfamily\descitemfont{Optima}
%\newfontfamily\notefont{Lucida Handwriting}
\newcommand\descitem[1]{{\notefont #1}}
\newcommand\descitemcol[1]{{\color{headings}\descitemfont #1}}

\newenvironment{vlist}[1]{%
\begin{list}{}{%
    \settowidth{\labelwidth}{\tt #1 }     %longest label length
%    \addtolength{\labelsep}{3ex}          %extra space between lbl/txt
    \setlength{\leftmargin}{\labelwidth}  %determine leftmargin
    \addtolength{\leftmargin}{\labelsep}  %determine leftmargin
    \setlength{\parsep}{0.5ex plus 0.2ex minus 0.2ex}
    \setlength{\itemsep}{0.3ex}
    \renewcommand{\makelabel}[1]{\color{headings}\tt ##1 \color{text1}\hfill}}}%
{\end{list}}
%

\usepackage{setspace}
%\onehalfspacing
%\setstretch{1.16}%onehalf is 1.25/1.213/1.241 for 10/11/12 pt
\setstretch{1.10}

\usepackage{titlesec} % Allows creating custom \section's
% Format of the section titles
\titleformat{\section}{\color{headings}
%\scshape%
\descitemfont
\Huge\raggedright}{}{0em}{}[\color{black}\titlerule]
\titlespacing{\section}{0pt}{0pt}{5pt} % Spacing around titles

%\newfontfamily\subsubsectionfont[Color=MSLightBlue]{Times New Roman}
% Set formats for each heading level
%\titleformat*{\section}{\Large\bfseries\sffamily\color{MSBlue}}
%\titleformat*{\subsection}{\large\bfseries\sffamily\color{MSLightBlue}}
\titleformat*{\subsection}{\color{headings}}


\usepackage{enumitem}
\SetLabelAlign{parright}{\parbox[t]{\labelwidth}{\raggedleft#1}}
\setlist[description]{style=multiline,topsep=10pt,leftmargin=2.5cm,font=\normalfont,%
    align=parright}

\newlength{\pdfwidth}
%%\setlength{\pdfwidth}{1.0\textwidth}
\setlength{\pdfwidth}{0.45\textwidth}
\newlength{\halfpage}
\setlength{\halfpage}{0.5\textwidth}
\newlength{\halfpagefig}
\setlength{\halfpagefig}{0.5\textwidth}
\addtolength{\halfpagefig}{-0.25cm}

\newcommand{\fit}[1]{\noindent\resizebox{\linewidth}{!}{#1}}

\usepackage{comment}

\begin{document}

\section*{\Huge{\textbf{X-Smoke}}}
\vspace{1cm}

Plugin that draws two smoke trails behind your plane.

\vspace{0.5\baselineskip}
It needs X-Plane 11 (64 bits only), and works on Linux, Mac
and Windows.

\vspace{0.5\baselineskip}
The smoke consists of simple smoke puffs, which linger a
while before they disappear. I wrote this to enhance my acrobatic
flying, but it can also be used to create contrails, to
simulate crop dusting or to find the multi-player (AI) planes.

A number of different smoke types have been included. Some are quite
static, and are meant to show the flight path more than create a
realist smoke effect. The smoke effect for the multi-player planes is
fixed at the moment.

\vspace{0.5\baselineskip}
The smoke can be started and stopped in two ways. First, from the
menu, and second, by binding a key or joystick button to the custom command
defined by the plugin. The command is called
\texttt{durian/xsmoke/toggle}. 

\vspace{1cm}
\section*{\Huge{\textbf{Installation}}}
\vspace{1cm}

The plugin is distributed as a zip file. The \texttt{xsmoke} directory
and its contents, contained in the zip file, must be put in the
\texttt{Resources/plugins} directory which can be found in the main
X-Plane directory. The directory containing the plugin \textsl{must}
be called \texttt{xsmoke}. If installed correctly, a new menu item
called ``X-Smoke'' will appear under the Plugins menu in X-Plane. This
menu contains a number of items. The first one, called ``Toggle
Smoke'', switches the smoke on or off. The second one, called ``Toggle
AI smoke'', puts a smoke trail behind the multi-player planes (the
ones that are moving). The third one, called ``Rescan directory'',
rescans the directory for new ini files and removes the smoke
currently in the air. It also resets the selected smoke to the one
called \texttt{smoke.ini}.

These three menu items are followed by a list of initialisation files.
The file called \texttt{smoke.ini} is loaded on start-up. You can
select another initialisation file whenever you want. The menu lists
all files with suffix \texttt{ini}, and you can add new ones if you
like. 

If you put a file called \texttt{smoke.ini} in an aircrafts directory
(where the \texttt{acf} file resides), it will override the
\texttt{smoke.ini} in the plugin directory. This allows you to
automatically load the desired kind of smoke per aircraft.

\vspace{0.5\baselineskip}
The initialisation files contain a number of parameters you can
change. The default settings produce a white vapour trail. 

\vspace{3cm}
\section*{\Huge{\textbf{Parameters}}}
\vspace{1cm}

\vspace{0.5\baselineskip}
You can tweak the smoke colour and transparency to make the smoke look the
way you want. Select the relevant entry from the menu again after you
have made a change. This will reload the settings.

The puffs have two phases. First they grow from their minimum till
maximum size (the grow phase), and then they linger (the linger
phase). When their linger time is up, they disappear. 

In the grow phase, the size, speed and transparency of the smoke puffs
are changed according to the parameters. In the linger phase, these
parameters stay fixed, but the smoke puffs can still grow (or
diminish) in size.

If the puffs hit the ground, they stop their vertical movement. The
will still grow and move with the wind.

There are (too) many setting to control the behaviour of the smoke
puffs. They have been sorted into four groups containing related
settings.

\vspace{0.2cm}
\subsection*{\large{\textbf{General}}}
\vspace{0.2cm}

\begin{vlist}{}

\item[smoke\_max = 50000] The maximum number of smoke puffs drawn by
  the plugin. When the maximum is reached, the generation of puffs is
  suspended until enough smoke puffs have disappeared. \texttt{50000}
  is quite a high number which you will probably not reach.

\item[smoke\_refresh = -1] Specifies how often the smoke draw routine is called.
  Leave this at \texttt{-1} to draw them every frame. To run the smoke
  engine 20 times per second, specify \texttt{0.05}.

\end{vlist}{}

\vspace{0.2cm}
\subsection*{\large{\textbf{Smoke emitters}}}
\vspace{0.2cm}

There are two ways to specify the locations of where the smoke is
emitted. The first example shows the easiers method; the plugin scans
the plane to determine where the engines are. The second example shows
how they can be specified manually.

\begin{vlist}{}

\item[plane\_smoke\_emitter = 0] This is the easiest way to specify
  the locations where the smoke is emitted. The plugin will scan the
  plane for the positions of the engines. Each engine is assigned an
  emitter. It is still possible to specify offsets (see
  \texttt{plane\_smoke\_emit\_offset} below to adjust the position).

\item[plane\_smoke\_emitter = 10,20,10] If you specify three
  parameters, the engine locations are not scanned. Instead, these three numbers are
  parameters for the loop that determines where the smoke puffs are emitted.
  The example values will generate 10 and 20. This means that
  smoke is emitted 10 meters to the right from the centre of the plane, and at 20
  meters. The points are mirrored, so this example specifies 4 points where
  smoke is emitted, approximately where the 4 engines of a 747 are.
  \par\indent The loop starts at the first value specified, \texttt{10}. The
  increment is specified in the third value, \texttt{10}, so the next
  value is \texttt{20}. The end value, the second value, is \texttt{20}

\item[plane\_smoke\_emit\_offset = 0.5, -0.5, 2.0] This setting
  adjusts how far to the right, below and behind the calculated or
  specified emitter location smoke is emitted. The first value in the
  example specifies \texttt{0.5} meters to the right of the location,
  the second one \texttt{0.5} meters \textsl{under} the location, and
  the third one \texttt{2} meters \textsl{behind} the location. The
  offset is applied to each emitter point. The locations obtained
  automatically are often inside the engines, and with this parameter
  they can be moved to a more suitable location.

\item[plane\_smoke\_emit\_rate = 120] The number of smoke puffs 
  generated per second. Higher numbers means thicker smoke. Applied to
  each emitter point individually. A high rate coupled with a long
  linger time can cause the maximum smoke puffs to be reached easily.
  With a single engine plane, and emitting 120 puffs per second,  
  a \texttt{smoke\_max} setting of 20000 is reached in 166 seconds.
  With a twin engine plane, in half that time.

\item[plane\_smoke\_flow = 0.8] Smoke puffs are emitted in about 80\%
  of the time. Can be used to make the trail thinner or thicker.

\item[plane\_smoke\_throttle = 0] Setting this parameter to \texttt{1}
  varies the amount of smoke with the throttle setting. The
  \texttt{plane\_smoke\_emit\_rate}, \texttt{plane\_smoke\_flow} and
  \texttt{plane\_smoke\_emit\_velocity} are varied between
  \texttt{25}\% and \texttt{100}\% of the specified value depending on
  the throttle setting. It doesn't detect if the engine is running
  (yet), it only looks at the position of the throttle.

\end{vlist}{}

\vspace{0.2cm}
\subsection*{\large{\textbf{Smoke size and duration}}}
\vspace{0.2cm}

\begin{vlist}{}

\item[plane\_smoke\_size = 0.0, 5.0, 1.0] The smoke grows from the
  first number \texttt{0.0} to the second number \texttt{5.0}, with a rate specified by the
  third number. The rate in this example is \texttt{1.0} which means
  that the puff takes \texttt{5} seconds to grow to its maximum size. The puff
  then lingers in the air for the amount of time specified by the
  \texttt{plane\_smoke\_linger} parameter. The size values are meters,
  and the growth rate is in meters per second.

\item[plane\_smoke\_linger = 30, 60, 0.5] Smoke will linger between \texttt{30} plus
  a random number between \texttt{0} and \texttt{60} seconds, and grow in size with a
  rate of \texttt{0.5} m/s.

\end{vlist}{}

\vspace{0.2cm}
\subsection*{\large{\textbf{Smoke colours and transparency}}}
\vspace{0.2cm}

\begin{vlist}{}

\item[plane\_smoke\_rgb\_left = 255,255,255] The RGB colour of the smoke
  on the left side.

\item[plane\_smoke\_rgb\_right = 255,255,255] The RGB colour of the smoke
  on the right side.

\item[plane\_smoke\_colour\_variation = 0.4] The RGB values specified
  above are varied by \texttt{40}\%. This gives better looking smoke.

\item[plane\_smoke\_transparency = 0.2, 0.01] The transparency of the
  puffs. The transparency is varied from the first number \texttt{0.2} to the
  second number \texttt{0.01} over the age of the puff. The age is
  calculated from the \texttt{plane\_smoke\_size} parameter explained earlier.

\end{vlist}{}

\vspace{0.2cm}
\subsection*{\large{\textbf{Smoke movement}}}
\vspace{0.2cm}

\begin{vlist}{}

\item[plane\_smoke\_movement = 0.3] The initial speed of the puffs, in
  m/s. Higher speeds means the smoke disperses more. The speed
  assigned will be a random number less than the specified value. The
  puffs change their initial random speed and direction to the direction and
  speed of the wind in the grow phase. They keep moving at the speed
  of the wind in the linger phase.

\item[plane\_smoke\_emit\_velocity = 4.4] The exit velocity of the smoke
  in m/s.

\item[plane\_smoke\_vy = 2.0, -1.0] The desired vertical speed of the
  puffs at the start and end of the grow phase. They will continue to
  rise or fall in the linger phase at the speed reached at the end.
  The example has the smoke rise in the beginning, at \texttt{2} m/s,
  and then slow down and fall to \texttt{-1} m/s (that is, going down)
  during the grow phase.

\item[plane\_smoke\_wind\_factor = 1.0] The actual wind speed is multiplied
  by this factor when assigning it to the smoke. Set it to \texttt{0}
  to make the smoke impervious to the wind.

\end{vlist}

\vspace{0.2cm}
\subsection*{\large{\textbf{Version history}}}
\vspace{0.2cm}

%\vspace{0.5\baselineskip}
This readme is for version 1.1.1. 

Main changes over 1.0.0 are the change of the name to \texttt{xsmoke},
the addition of the multi-player smoke, and the X-Plane 11 only
requirement. Fixes for 11.41 and 11.50.

\vspace{0.5\baselineskip} {\color{text1}
 May not be re-distributed, sold or used for commercial purposes without explicit permission.}
\end{document}
